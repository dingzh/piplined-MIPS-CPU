\documentclass[a4paper]{article}
\usepackage[CJKaddspaces ,CJKchecksingle ,CJKnumber]{xeCJK}
    \setCJKmainfont[BoldFont={Adobe Heiti Std},
    ItalicFont={Adobe Kaiti Std}]{Adobe Song Std}
    \setCJKsansfont{Adobe Heiti Std}
    \setCJKmonofont{Adobe Fangsong Std}
    \punctstyle{hangmobanjiao}
\usepackage{booktabs}
\usepackage{listings}
\usepackage{xcolor}
\usepackage{underscore}


\lstset{ %
  backgroundcolor=\color{white},   % choose the background color; you must add \usepackage{color} or \usepackage{xcolor}
  basicstyle=\footnotesize,        % the size of the fonts that are used for the code
  breakatwhitespace=false,         % sets if automatic breaks should only happen at whitespace
  breaklines=true,                 % sets automatic line breaking
  captionpos=bl,                    % sets the caption-position to bottom
  deletekeywords={...},            % if you want to delete keywords from the given language
  escapeinside={\%*}{*)},          % if you want to add LaTeX within your code
  extendedchars=true,              % lets you use non-ASCII characters; for 8-bits encodings only, does not work with UTF-8
  frame=single,                    % adds a frame around the code
  keepspaces=true,                 % keeps spaces in text, useful for keeping indentation of code (possibly needs columns=flexible)
  keywordstyle=\color{blue},       % keyword style
  %language=Python,                 % the language of the code
  morekeywords={*,...},            % if you want to add more keywords to the set
  numbers=left,                    % where to put the line-numbers; possible values are (none, left, right)
  numbersep=5pt,                   % how far the line-numbers are from the code
  rulecolor=\color{black},         % if not set, the frame-color may be changed on line-breaks within not-black text (e.g. comments (green here))
  showspaces=false,                % show spaces everywhere adding particular underscores; it overrides 'showstringspaces'
  showstringspaces=false,          % underline spaces within strings only
  showtabs=false,                  % show tabs within strings adding particular underscores
  stepnumber=1,                    % the step between two line-numbers. If it's 1, each line will be numbered
  stringstyle=\color{orange},     % string literal style
  tabsize=2,                       % sets default tabsize to 2 spaces
}
\newcommand{\HRule}{\rule{\linewidth}{0.5mm}}

\begin{document}

\begin{titlepage} 
    \begin{center} % Upper part of the page 
    \includegraphics[width=0.15\textwidth]{sjtu.png}\\[1cm] 
    \textsc{\LARGE CS334 计算机组成实验}\\[1.5cm] 
    \textsc{\Large Lab6}\\[0.5cm] % Title 
    \HRule \\[0.4cm] { \huge \bfseries 简单的类 MIPS多周期流水化处理器}\\[0.4cm] 
    \HRule \\[1.5cm] % Author and supervisor 
    \begin{minipage}{0.4\textwidth} 
        \begin{flushleft} 
        \large 
        周鼎\\
        5140219268
        \end{flushleft} 
        \end{minipage} 
        \begin{minipage}{0.4\textwidth} 
        \begin{flushright} 
        \large 
        指导老师:王老师
        \end{flushright} 
        \end{minipage} 
        \vfill % Bottom of the page 
        {\large \today} 
    \end{center} 
\end{titlepage}


\newpage
\tableofcontents

\newpage
\part{基本的流水线实现}
\section{五级流水线的顶层结构}
将指令执行分为五个阶段,每个阶段每条指令只使用一个主要模块,如下图所示,相比单周期,每阶段间添加寄存器存储必要的信息。
\begin{figure}[h]
  \centering
  \includegraphics[width=0.95\textwidth]{top.pdf}
  \caption{顶层设计}\label{TOP}
\end{figure}
\section{系统定义顶层所需的寄存器}
流水线的顶层中需要较多的寄存器,为了做到不重不漏,按阶段和用途分类定义。分为IF/ID, ID/EX, EX/MEM, MEM/WB四个大块,内分为控制信号和指令信息,指令信息里有PC+4,也有后续指令执行过程得到的结果;控制信号按信号使用的阶段分类,具体见如下代码。
\begin{lstlisting}[language={Verilog},title={reg_def_top.v}]
    //1.0 For stage IF to ID;
    reg [31:0] IF_ID_PCAdd4;
    reg [31:0] IF_ID_Instr;

    //2.0 For stage ID to EX;
    reg [31:0]  ID_EX_PCAdd4;
    reg [31:0]  ID_EX_RegReadData1;
    reg [31:0]  ID_EX_RegReadData2;
    reg [31:0]  ID_EX_SignExt;
    reg [20:16] ID_EX_InstHigh;
    reg [15:11] ID_EX_InstLow;
    //2.1 To EX
    reg ID_EX_RegDst;
    reg [1:0] ID_EX_ALUOp;
    reg       ID_EX_ALUSrc;
    //2.2 To MEM
    reg ID_EX_Branch;
    reg ID_EX_MemWrite;
    reg ID_EX_MemRead;
    //2.3 To WB
    reg ID_EX_MemToReg;
    reg ID_EX_RegWrite;

    //3.0 For stage EX to MEM;
    reg        EX_MEM_Zero;
    reg [31:0] EX_MEM_ALUOut;
    reg [31:0] EX_MEM_BranchAddress;
    reg [31:0] EX_MEM_RegReadData2;
    //3.1 To MEM
    reg EX_MEM_Branch;					
    reg EX_MEM_MemWrite;
    reg EX_MEM_MemRead;
    reg [4:0]  EX_MEM_RegWriteAddress;  //rt or rd
    //3.2 To WB
    reg EX_MEM_MemToReg;
    reg EX_MEM_RegWrite;

    //4.0 For stage MEM to WB;
    reg [31:0] MEM_WB_ALUOut;
    reg [31:0] MEM_WB_MemReadData;
    reg [4:0]  MEM_WB_RegWriteAddress;
    reg        MEM_WB_RegWrite;
    reg        MEM_WB_MemToReg;   \end{lstlisting}
\section{分阶段定义网线类型并完成实例化和连接}
    \subsection{IF Stage}
    IF阶段的实现要特别注意线型变量不要和WB阶段混淆,以及NEXT_PC的实现。
        \begin{lstlisting}[language={Verilog},title={IF.v}]
//1.0 IF
	wire IF_PCSrc,  //for MUX sel signal
	     IF_Branch,
		  IF_Zero;
	wire [31:0] IF_BranchAddress;
	wire [31:0] IF_CurrPC;
	wire [31:0] IF_NextPC;
	wire [31:0] IF_PCAdd4;
	wire [31:0] IF_Instr;	
	//associate with reg
	assign IF_BranchAddress = EX_MEM_BranchAddress;
	assign IF_Zero = EX_MEM_Zero;
	assign IF_Branch = EX_MEM_Branch;
	//Combinational Logic
	assign IF_PCAdd4 = IF_CurrPC +4;
	assign IF_PCSrc  = IF_Branch & IF_Zero & ~RESET;
	assign IF_NextPC = IF_PCSrc? IF_BranchAddress: IF_PCAdd4;  //MUX
	//assign IF_NextPC = IF_PCAdd4;
	//update PC at posedge
	Pc mainPC (
    .clock_in(CLK),
    .nextPC(IF_NextPC),
    .currPC(IF_CurrPC),
    .rst(RESET)
    );
	wire [31:0] IF_Index;
	assign IF_Index = IF_CurrPC>>2;
	instructionMemory InstrMemory (
    .address(IF_Index),
    .clock_in(CLK),
    .reset(RESET),
    .readData(IF_Instr)
    );
	//InstMem;
   \end{lstlisting}
    \subsection{ID Stage}
    ID阶段所需要的mainCtr与单周期的功能完全一样,课直接使用,唯一不同是产生的控制信号需要用寄存器存储至相应阶段使用。
        \begin{figure}[hb]
  \centering
  \includegraphics[width=0.8\textwidth]{ctrsig.pdf}
  \caption{后三阶段的控制信号}
\end{figure}
        \begin{lstlisting}[language={Verilog},title={ID.v}] wire [31:0] ID_Instr;
	wire [5:0] ID_OpCode;
	wire [31:0] ID_PCAdd4;
	wire [4:0] ID_RegReadAddress1;
	wire [4:0] ID_RegReadAddress2;
	wire [15:0] ID_Imm;
	wire [31:0] ID_SignExt;
	wire [20:16] ID_InstHigh;
	wire [15:11] ID_InstLow;
	wire [31:0] ID_RegReadData1;
	wire [31:0] ID_RegReadData2;
	wire WB_RegWrite;
	wire WB_MemToReg;
	wire [4:0]  WB_RegWriteAddress;
	wire [31:0] WB_MemReadData;
	wire [31:0] WB_ALUOut;
	wire [31:0] WB_RegWriteData;
	//associate with reg
	assign ID_Instr = IF_ID_Instr;
	assign ID_PCAdd4 = IF_ID_PCAdd4;
	assign ID_OpCode = ID_Instr[31:26];
	assign ID_RegReadAddress1 = ID_Instr[25:21];
	assign ID_RegReadAddress2 = ID_Instr[20:16];
	assign ID_Imm = ID_Instr[15:0];
	assign ID_InstHigh = ID_Instr[20:16];
	assign ID_InstLow  = ID_Instr[15:11];
	assign WB_RegWrite = MEM_WB_RegWrite;
	assign WB_MemToReg = MEM_WB_MemToReg;
	assign WB_RegWriteAddress = MEM_WB_RegWriteAddress;
	assign WB_ALUOut = MEM_WB_ALUOut;
	assign WB_MemReadData = MEM_WB_MemReadData;
	assign WB_RegWriteData = WB_MemToReg? WB_MemReadData: WB_ALUOut;
	
	Register mainReg (
    .clock_in(CLK),
    .regWrite(WB_RegWrite),
    .readReg1(ID_RegReadAddress1),
    .readReg2(ID_RegReadAddress2),
    .writeReg(WB_RegWriteAddress),
    .writeData(WB_RegWriteData),
    .reset(RESET),
    .readData1(ID_RegReadData1),
    .readData2(ID_RegReadData2),
	 .ioinput(SW),
	 .iooutput(LED)
    );
	
	 //2.1 To EX
	 wire [1:0] ID_ALUOp;
	 wire ID_RegDst;
	 wire ID_ALUSrc;
	 //2.2 To MEM
	 wire ID_Branch;
	 wire ID_MemWrite;
	 wire ID_MemRead;
	 //2.3 To WB
	 wire ID_MemToReg;
	 wire ID_RegWrite;
	 wire JUMP;  //of no use
	 Ctr mainCtr (
    .opCode(ID_OpCode),
    .regDst(ID_RegDst),
    .aluSrc(ID_ALUSrc),
    .memToReg(ID_MemToReg),
    .regWrite(ID_RegWrite),
    .memRead(ID_MemRead),
    .memWrite(ID_MemWrite),
    .branch(ID_Branch),
    .aluOp(ID_ALUOp),
    .jump(JUMP)
    );	
	 signExt mainSignExt (
    .inst(ID_Imm),
    .data(ID_SignExt)
    );  \end{lstlisting}
    \subsection{EX \& MEM}
    后面的过程与单周期十分类似,不在赘述,仅列出代码。
        \begin{lstlisting}[language={Verilog},title={EX.v}]
	//2.0 For stage ID to EX;
	wire [31:0] EX_PCAdd4;
	wire [31:0] EX_ALUSrc1;
	wire [31:0] EX_ALUSrc2;
	wire [31:0] EX_RegReadData2;
	wire [31:0] EX_SignExt;
	wire [20:16] EX_InstHigh;
	wire [15:11] EX_InstLow;
	wire EX_RegDst,
	     EX_ALUSrc;
	wire [1:0] EX_ALUOp;
	wire [4:0] EX_RegWriteAddress;
	wire [5:0] EX_Funct;	
	//associate with Reg
	assign EX_RegDst  = ID_EX_RegDst;
	assign EX_ALUOp   = ID_EX_ALUOp;
	assign EX_ALUSrc1 = ID_EX_RegReadData1;
	assign EX_ALUSrc  = ID_EX_ALUSrc;     //contral signal
	assign EX_ALUSrc2 = EX_ALUSrc? EX_SignExt: EX_RegReadData2;    //MUX
	assign EX_RegWriteAddress = EX_RegDst? EX_InstLow: EX_InstHigh;
	assign EX_PCAdd4 = ID_EX_PCAdd4;	
	assign EX_RegReadData2 = ID_EX_RegReadData2;
	assign EX_SignExt  = ID_EX_SignExt;
	assign EX_Funct    = ID_EX_SignExt[5:0];
	assign EX_InstHigh = ID_EX_InstHigh;
	assign EX_InstLow  = ID_EX_InstLow;
	//Instances	
	wire [3:0]   EX_ALUCtr;
	AluCtr mainALUCtr (
    .aluOp(EX_ALUOp),
    .funct(EX_Funct),
    .aluCtr(EX_ALUCtr)
    );	
	wire EX_Zero;
	wire [31:0] EX_ALUOut;	
	Alu mainALU (
    .input1(EX_ALUSrc1),
    .input2(EX_ALUSrc2),
    .aluCtr(EX_ALUCtr),
    .zero(EX_Zero),
    .aluRes(EX_ALUOut)
    );
	 //2.2 To MEM
	 wire EX_Branch,
			EX_MemWrite,
			EX_MemRead;
	 assign EX_Branch   = ID_EX_Branch;
	 assign EX_MemWrite = ID_EX_MemWrite;
	 assign EX_MemRead  = ID_EX_MemRead;
	 //2.3 To WB
	 wire EX_MemToReg,
			EX_RegWrite;
	 assign EX_MemToReg = ID_EX_MemToReg;
	 assign EX_RegWrite = ID_EX_RegWrite;
	 //BranchAddress
	 wire [31:0] EX_BranchAddress;
	 assign EX_BranchAddress = (EX_SignExt<<2) + EX_PCAdd4;   \end{lstlisting}

        \begin{lstlisting}[language={Verilog},title={MEM.v}]
//4.0 MEM
	wire MEM_MemWrite,
		  MEM_MemRead;
	wire [4:0] MEM_RegWriteAddress;
	wire [31:0] MEM_ALUOut;
	wire [31:0] MEM_MemWriteData;
	wire [31:0] MEM_MemReadData;
	//associate with reg
	assign MEM_MemWrite = EX_MEM_MemWrite;
	assign MEM_MemRead = EX_MEM_MemRead;
	assign MEM_RegWriteAddress = EX_MEM_RegWriteAddress;
	assign MEM_ALUOut = EX_MEM_ALUOut;
	assign MEM_MemWriteData = EX_MEM_RegReadData2;
	//instances
   dataMemory DataMemory (
    .clock_in(CLK),
    .address(MEM_ALUOut),
    .writeData(MEM_MemWriteData),
    .readData(MEM_MemReadData),
    .memWrite(MEM_MemWrite),
    .memRead(MEM_MemRead)
    );	
	 //to WB
	 wire MEM_MemToReg,
			MEM_RegWrite;
	 assign MEM_MemToReg = EX_MEM_MemToReg;
	 assign MEM_RegWrite = EX_MEM_RegWrite;   \end{lstlisting}




    \subsection{阶段间的寄存器更新}
        每个阶段之间的寄存器在上升沿更新,
        \begin{lstlisting}[language={Verilog},title={internalRegUpdate.v}]
//1.5 IF/ID REG UPDATE
	always @(posedge CLK)	begin
			IF_ID_PCAdd4 <= IF_PCAdd4;
			IF_ID_Instr <= IF_Instr;
		end
//2.5 ID/EX REG UPDATE
	always @(posedge CLK)	begin
		  //2.0 For Stage ID To EX
	     ID_EX_PCAdd4       <= ID_PCAdd4;
	     ID_EX_RegReadData1 <= ID_RegReadData1;
	     ID_EX_RegReadData2 <= ID_RegReadData2;
	     ID_EX_SignExt      <= ID_SignExt;
	     ID_EX_InstHigh     <= ID_InstHigh;
	     ID_EX_InstLow      <= ID_InstLow;
	     //2.1 To EX
	     ID_EX_RegDst       <= ID_RegDst;
	     ID_EX_ALUOp        <= ID_ALUOp;
	     ID_EX_ALUSrc       <= ID_ALUSrc;
	     //2.2 To MEM
	     ID_EX_Branch       <= ID_Branch;
	     ID_EX_MemWrite     <= ID_MemWrite;
	     ID_EX_MemRead      <= ID_MemRead;
	     //2.3 To WB
	     ID_EX_MemToReg     <= ID_MemToReg;
	     ID_EX_RegWrite     <= ID_RegWrite;
		end
//3.5 EX/MEM REG UPDATE
   always @(posedge CLK)  begin
		 EX_MEM_Branch <= EX_Branch;
		 EX_MEM_MemWrite <= EX_MemWrite;
		 EX_MEM_MemRead <= EX_MemRead;
		 EX_MEM_MemToReg <= EX_MemToReg;
		 EX_MEM_RegWrite <= EX_RegWrite;
		 EX_MEM_BranchAddress <= EX_BranchAddress;
		 EX_MEM_Zero <= EX_Zero;
		 EX_MEM_ALUOut <= EX_ALUOut;
		 EX_MEM_RegWriteAddress <= EX_RegWriteAddress;
		 EX_MEM_RegReadData2 <= EX_RegReadData2;
	  end	
//4.5 MEM/WB REG UPDATE
	always @(posedge CLK)	begin
	      MEM_WB_ALUOut = MEM_ALUOut;
	      MEM_WB_MemReadData = MEM_MemReadData;
	      MEM_WB_RegWriteAddress = MEM_RegWriteAddress;
	      MEM_WB_RegWrite = MEM_RegWrite;
	      MEM_WB_MemToReg = MEM_MemToReg;
      end   \end{lstlisting}

\section{ModelSim仿真}
%仿真程序和仿真结果见下图,可以正常运行。
    \begin{figure}[h]
        \begin{minipage}[t]{0.54\linewidth}
        \centering
            \begin{lstlisting}[language={Verilog}]
10001100000000010000000000101100
10001100000000100000000000110000
10001100000000110000000000110100
00000000000000000000000000000000
00000000000000000000000000000000
00000000001000100010000000100000
00000000011000010010100000100010
00010000000000000000000000000100
00000000000000000000000000000000
00000000000000000000000000000000
00000000000000000000000000000000
00000000010000110011000000100000
10001100000001110000000000101000
00010000000000001111111111111111
00000000000000000000000000000000
00000000000000000000000000000000  \end{lstlisting}
        \caption{二进制指令}
        \end{minipage}%
        \begin{minipage}[t]{0.46\linewidth}
        \centering
            \begin{lstlisting}[language={Verilog}]
lw $1, 44($0)  ; 2
lw $2, 48($0)  ; 3
lw $3, 52($0)  ; 4
NOP
NOP
add $4, $1, $2  ; $4=5
sub $5, $3, $1  ; $5=2
beq $0, $0, end ;
NOP
NOP
NOP
add $6, $2, $3  ; not executed
end:    lw $7, 40($0)  ; 1
beq $0, $0, -1 ;
NOP
NOP   \end{lstlisting}
        \caption{MIPS指令}
        \end{minipage}
    \end{figure}

    \begin{figure}[h]
        \centering
        \includegraphics[width=1\textwidth]{topsimori.png}
        \caption{ModelSim仿真}
    \end{figure}


\newpage
\part{改进}
实现的最基本的流水线已经能够运行,但在测试程序时很容易发现问题:\\
    \begin{enumerate}
      \item BEQ指令后必须插入NOP,这在beq不跳转时会浪费时钟周期;
      \item 没有实现jump指令;
      \item 没有解决data hazard, 必须插入NOP避免read-after-write;
    \end{enumerate}
    下面将着手解决这些问题。
    \section{Flush}
    \subsection{改进方案}
    执行beq指令时,当IF_Branch有效之后,下一上升沿将load跳转后的指令,此时已经执行到ID,EX阶段的指令需要Flush掉,因此ID/EX, EX/MEM, MEM/WB的所有寄存器清零。下面是修改后的regUpdate代码。
     \begin{lstlisting}[language={Verilog},title={regUpdate.v}]
//1.5 IF/ID REG UPDATE
always @(posedge CLK)	begin
			IF_ID_PCAdd4 <= IF_Branch ? 0: IF_PCAdd4;
			IF_ID_Instr  <= IF_Branch ? 0: IF_Instr;
		end
//2.5 ID/EX REG UPDATE
always @(posedge CLK)	begin
	     //2.0 For Stage ID To EX
	     ID_EX_PCAdd4       <= IF_Branch ? 0: ID_PCAdd4;
	     ID_EX_RegReadData1 <= IF_Branch ? 0: ID_RegReadData1;
	     ID_EX_RegReadData2 <= IF_Branch ? 0: ID_RegReadData2;
	     ID_EX_SignExt      <= IF_Branch ? 0: ID_SignExt;
	     ID_EX_InstHigh     <= IF_Branch ? 0: ID_InstHigh;
	     ID_EX_InstLow      <= IF_Branch ? 0: ID_InstLow;
	     //2.1 To EX
	     ID_EX_RegDst       <= IF_Branch ? 0: ID_RegDst;
	     ID_EX_ALUOp        <= IF_Branch ? 0: ID_ALUOp;
	     ID_EX_ALUSrc       <= IF_Branch ? 0: ID_ALUSrc;
	     //2.2 To MEM
	     ID_EX_Branch       <= IF_Branch ? 0: ID_Branch;
	     ID_EX_MemWrite     <= IF_Branch ? 0: ID_MemWrite;
	     ID_EX_MemRead      <= IF_Branch ? 0: ID_MemRead;
	     //2.3 To WB
	     ID_EX_MemToReg     <= IF_Branch ? 0: ID_MemToReg;
	     ID_EX_RegWrite     <= IF_Branch ? 0: ID_RegWrite;
		end
//3.5 EX/MEM REG UPDATE
always @(posedge CLK)  begin
        EX_MEM_Branch <= IF_Branch ? 0: EX_Branch;
        EX_MEM_MemWrite <= IF_Branch ? 0: EX_MemWrite;
        EX_MEM_MemRead <= IF_Branch ? 0: EX_MemRead;
        EX_MEM_MemToReg <= IF_Branch ? 0: EX_MemToReg;
        EX_MEM_RegWrite <= IF_Branch ? 0: EX_RegWrite;
        EX_MEM_BranchAddress <= IF_Branch ? 0: EX_BranchAddress;
        EX_MEM_Zero <= IF_Branch ? 0: EX_Zero;
        EX_MEM_ALUOut <= IF_Branch ? 0: EX_ALUOut;
        EX_MEM_RegWriteAddress <= IF_Branch ? 0: EX_RegWriteAddress;
        EX_MEM_RegReadData2 <= IF_Branch ? 0: EX_RegReadData2;
	  end  \end{lstlisting}

    \subsection{ModelSim仿真}
\begin{figure}[h]
  \centering
  \includegraphics[width=0.92\textwidth]{jumpsim.png}
  \caption{ModelSim Flush仿真}
\end{figure}

        \begin{figure}[h]
        \begin{minipage}[t]{0.54\linewidth}
        \centering
            \begin{lstlisting}[language={Verilog}]
00000000000000000000000000000000
10001100000000010000000000101100
10001100000000100000000000110000
10001100000000110000000000110100
00000000000000000000000000000000
00000000000000000000000000000000
00000000001000100010000000100000
00000000011000010010100000100010
00010000000000000000000000000001
00000000010000110011000000100000
10001100000001110000000000101000
00010000000000001111111111111111  \end{lstlisting}
        \caption{二进制指令}
        \end{minipage}%
        \begin{minipage}[t]{0.46\linewidth}
        \centering
            \begin{lstlisting}[language={Verilog}]
NOP
lw $1, 44($0)  ; 2
lw $2, 48($0)  ; 3
lw $3, 52($0)  ; 4
NOP
NOP
add $4, $1, $2  ; $4=5
sub $5, $3, $1  ; $5=2
beq $0, $0, end ;
add $6, $2, $3  ; not executed
end:    lw $7, 40($0)  ; 1
beq $0, $0, -1 ;   \end{lstlisting}
        \caption{MIPS指令}
        \end{minipage}
    \end{figure}

    \section{JUMP}
    \subsection{改进方案}
    实现jump需要生成跳转地址,给nextPC添加一个source,为了避免jump和beq指令的冲突,将jump指令延长至MEM阶段完成,同时jump也需要Flush功能。
    \begin{lstlisting}[language={Verilog},title{jump.v}]
\\Def Regs and Wires
	 reg ID_EX_Jump;
	 reg [31:0] ID_EX_Instr;
	 reg EX_MEM_Jump;
	 reg [31:0] EX_MEM_JumpAddress;
	 wire ID_Jump;
	 wire EX_Jump;
	 wire [31:0] EX_JumpAddress;
	 wire IF_Jump;
	 wire [31:0] IF_JumpAdress;
	 assign EX_Jump = ID_EX_Jump;
	 assign IF_Jump = EX_MEM_Jump;
	 assign EX_Instr = ID_EX_Instr;
	 assign EX_JumpAddress[27:2] = EX_Instr[25:0];
	 assign EX_JumpAddress[1:0] = 2'b00;
	 assign IF_JumpAddress = EX_MEM_JumpAddress;
\\Update Regs
	 always @(posedge CLK)	begin
		ID_EX_Jump <= ID_Jump;
		ID_EX_Instr <= ID_Instr;
		EX_MEM_Jump <= EX_Jump;
		EX_MEM_JumpAddress <= EX_JumpAddress;
	 end
\\Update PC
	assign IF_PCAdd4 = IF_CurrPC +4;
	assign IF_PCSrc  = IF_Branch & IF_Zero & ~RESET;
	wire [31:0] IF_NextPCJ;
	assign IF_NextPCJ = IF_Jump? IF_JumpAddress : IF_PCAdd4;
	assign IF_NextPC = IF_PCSrc? IF_BranchAddress: IF_NextPCJ;  //MUX  \end{lstlisting}
    \subsection{ModelSim仿真}
        \begin{figure}[h]
        \begin{minipage}[t]{0.54\linewidth}
        \centering
            \begin{lstlisting}[language={Verilog}]
00001000000000000000000000000110
00010000000000001111111111111111
00010000000000001111111111111111
00010000000000001111111111111111
00010000000000001111111111111111
10001100000000010000000000101100
10001100000000100000000000110000
10001100000000110000000000110100
00000000000000000000000000000000
00000000000000000000000000000000
00000000001000100010000000100000
00000000011000010010100000100010
00010000000000000000000000000001
00000000010000110011000000100000
10001100000001110000000000101000
00010000000000001111111111111111  \end{lstlisting}
        \caption{二进制指令}
        \end{minipage}%
        \begin{minipage}[t]{0.46\linewidth}
        \centering
            \begin{lstlisting}[language={Verilog}]
J 6;
circle1:    beq $0, $0, circle1 ;
circle2:    beq $0, $0, circle2 ;
circle3:    beq $0, $0, circle3 ;
circle4:    beq $0, $0, circle4 ;
lw $1, 44($0)  ; 2
lw $2, 48($0)  ; 3
lw $3, 52($0)  ; 4
NOP
NOP
add $4, $1, $2  ; $4=5
sub $5, $3, $1  ; $5=2
beq $0, $0, end ;
add $6, $2, $3  ; not executed
end:    lw $7, 40($0)  ; 1
beq $0, $0, -1 ;   \end{lstlisting}
        \caption{MIPS指令}
        \end{minipage}
    \end{figure}
           \begin{figure}
            \centering
                \includegraphics[width=1\textwidth]{topflush.png}
            \caption{ModelSim JUMP仿真}
        \end{figure}


    \section{数据冒险}
\begin{figure}[h]
  \centering
  \includegraphics[width=1\textwidth]{forwardtop.pdf}
  \caption{top with Forward Unit}
  \end{figure}


  \subsection{MUX}
  对ALU的两个输入增加两个MUX,其输入分别由信号ForwardA, ForwardB控制,控制信号对应的输入如下表所示;
\begin{figure}[h]
  \centering
  \includegraphics[width=0.8\textwidth]{MUXtable.pdf}
  \caption{MUX控制信号}
\end{figure}
将其模块化,代码如下
\begin{lstlisting}[language={Verilog},title={forwardMUX.v}]
module forwardMux(
    input [31:0] ID_EX,
    input [31:0] EX_MEM,
    input [31:0] MEM_WB,
    input [1:0] Forward,
    output [31:0] Sel
    );
	 wire [31:0] TEMP;
	 assign Sel =  Forward[1] ? EX_MEM : TEMP;
	 assign TEMP=	Forward[0] ? MEM_WB : ID_EX;
endmodule  \end{lstlisting}
    \subsection{Forward Unit}
    对于R-type指令引起的read-after-write, 希望通过如下图所示的方案解决数据冒险。创建模块ForwardUnit产生forwardMux的控制信号,实现图示的旁路。
    \begin{figure}[h]
  \centering
  \includegraphics[width=0.87\textwidth]{sol.pdf}
  \caption{Forwarding}
  旁路Forward逻辑很直观,见代码内部,不再单独列出。
\end{figure}
    \begin{lstlisting}[language={Verilog},title={forwardUnit.v}]
module forwardUnit(
    input [4:0] rs,
    input [4:0] rt,
    input MEM_WB_regWrite,
    input [4:0] MEM_WB_rd,
    input EX_MEM_regWrite,
    input [4:0] EX_MEM_rd,
    output reg [1:0] forwardA,
    output reg [1:0] forwardB,
	input rst
    );
	 always @(*)	begin
	 if(rst | MEM_WB_rd = 0 | EX_MEM_rd = 0)	begin
		forwardA = 'b00;
		forwardB = 'b00;
		end
	   else	begin
		if(MEM_WB_regWrite & MEM_WB_rd == rs)  	forwardA = 'b01;
		else if(EX_MEM_regWrite & EX_MEM_rd == rs)
                                                forwardA = 'b10;
			else forwardA = 'b00;

		if(MEM_WB_regWrite  & MEM_WB_rd == rt)	forwardB = 'b01;
		else if(EX_MEM_regWrite  & EX_MEM_rd == rt)
                                                forwardB = 'b10;
			else forwardB = 'b00;
		end
	 end
endmodule  \end{lstlisting}
    \subsection{连接信号线及实例化}
    如顶层的信号连接所示,连接相应的信号。
    \begin{lstlisting}[language={Verilog},title={forwardMUX.v}]
//FORWARDING PART
//MUX
	//wire [31:0] EX_ALUSrc1_f;		//as mux output;
	//wire [31:0] EX_ALUSrc2_f;		//defined before ALU
	wire [31:0] EX_MEM_ALU;		//input from other stage
	wire [31:0] MEM_WB_ALU;
	assign EX_MEM_ALU = EX_MEM_ALUOut;
	assign MEM_WB_ALU = MEM_WB_ALUOut;
	wire [1:0] forwardA;
	wire [1:0] forwardB;

//FORWARD UNIT	
	 wire [4:0] EX_MEM_regWriteAddress_f;
	 wire [4:0] MEM_WB_regWriteAddress_f;
	 wire	MEM_WB_regWrite_f;
	 wire EX_MEM_regWrite_f;
	 wire [4:0] rs_f;
	 wire [4:0] rt_f;
	 assign EX_MEM_regWriteAddress_f = EX_MEM_RegWriteAddress;
	 assign MEM_WB_regWriteAddress_f = MEM_WB_RegWriteAddress;
	 assign EX_MEM_regWrite_f = EX_MEM_RegWrite;
	 assign MEM_WB_regWrite_f = MEM_WB_RegWrite;
	 assign rs_f = ID_EX_Instr[25:21];//ID_EX
	 assign rt_f = ID_EX_Instr[20:16];
//instances	
	 forwardUnit mainFUnit (
    .rs(rs_f),
    .rt(rt_f),
    .MEM_WB_regWrite(MEM_WB_regWrite_f),
    .MEM_WB_rd(MEM_WB_regWriteAddress_f),
    .EX_MEM_regWrite(EX_MEM_regWrite_f),
    .EX_MEM_rd(EX_MEM_regWriteAddress_f),
    .forwardA(forwardA),
    .forwardB(forwardB),
	 .rst(RESET)
    );
	forwardMux MUXA (
    .ID_EX(EX_ALUSrc1),
    .EX_MEM(EX_MEM_ALU),
    .MEM_WB(MEM_WB_ALU),
    .Forward(forwardA),
    .Sel(EX_ALUSrc1_f)
    );
	 forwardMux MUXB (
    .ID_EX(EX_ALUSrc2),
    .EX_MEM(EX_MEM_ALU),
    .MEM_WB(MEM_WB_ALU),
    .Forward(forwardB),
    .Sel(EX_ALUSrc2_f)
    );
endmodule  \end{lstlisting}

    \subsection{ModelSim 仿真}使用ModelSim进行仿真,可以看到data hazard已经解决。
\begin{figure}[h]
  \centering
  \includegraphics[width=1\textwidth]{pipelinesim.png}
  \caption{ModelSim Forwarding Unit}
\end{figure}
        \begin{figure}[h]
        \begin{minipage}[t]{0.54\linewidth}
        \centering
            \begin{lstlisting}[language={Verilog}]
00000000001000100000100000100000
00000000001000100001000000100000
00000000001000100000100000100000
00000000001000100001000000100000
00000000001000100000100000100000
00000000001000100001000000100000
......  \end{lstlisting}
        \caption{二进制指令}
        \end{minipage}%
        \begin{minipage}[t]{0.46\linewidth}
        \centering
            \begin{lstlisting}[language={Verilog}]
add $1, $1, $2  ;
add $2, $1, $2  ;
add $1, $1, $2  ;
add $2, $1, $2  ;
add $1, $1, $2  ;
add $2, $1, $2  ;
......   \end{lstlisting}
        \caption{MIPS指令}
        \end{minipage}
    \end{figure}
\newline
\newline
\newline
\newline
\newline
\newline
\newline
\newline
\newline
\newline
\newline
\part{上板试验}
\section{IO Scheme}

    \begin{table}[h]
        \centering
        \begin{tabular}{ll}
              \toprule
              IO Port & Function\\
              \midrule
              LED[7]    &   查看clock信号 \\
              LED[6]    &   查看reset信号 \\
              LED[5:0]  &   查看寄存器数值 \\
              SW[3]     &   调整时钟快/慢模式 \\
              SW[2:0]   &   设置寄存器数值输出模式 \\
              Button    &   输入reste信号 \\
              \bottomrule
            \end{tabular}
            \caption{IO Scheme}
    \end{table}
        \begin{figure}[h]
        \begin{minipage}[t]{0.51\linewidth}
        \centering
        \begin{tabular}{cc}
          \toprule
          SW[2:0] & MODE\\
          % after \\: \hline or \cline{col1-col2} \cline{col3-col4} ...
          \midrule
          000 & led[5:3]为\$2,led[2:0]为\$1 \\
          001 & led显示\$1 \\
          011 & led显示\$2 \\
          111 & led显示PC \\
          \bottomrule
        \end{tabular}
        \end{minipage}%
        \begin{minipage}[t]{0.49\linewidth}
        \centering
        \begin{tabular}{l}
        \qquad    设计使用8个LED作为输出,\\
        4个开关和1个复位式按钮作为输\\
        入,具体功能见下图。因板上接口\\
        限制,将只对输出寄存器\$1,\$2和\\
        PC,通过SW[2:0]选择输出模式,\\
        见左图。\\
        \end{tabular}

        \end{minipage}
    \end{figure}

\newpage
\section{IO控制逻辑}
    \begin{lstlisting}[language={Verilog},title={IO.v}]
////            GENRERATING SLOW CLOCK          ////
	wire CLK;
	reg [26:0] Buffer = 0;
	always@ (posedge CLOCK) Buffer = Buffer + 1;
	assign CLK = FAST ? CLOCK : Buffer[26];
////            IO MODE SEL                     ////
	wire [31:0] IF_CurrPC;
	assign LED[7] = RESET;
	assign LED[6] = CLK;
	wire [5:0] OUTPUT;
	assign LED[5:0] = OUTPUT;
	wire [31:0] reg1;
	wire [31:0] reg2;
	wire [5:0]  reg12;
	assign reg12[5:3] = reg2;
	assign reg12[2:0] = reg1;
	wire [31:0] CURR_PC_IO;
	assign CURR_PC_IO = IF_CurrPC>>2;
	wire [5:0] TEMP1;
	wire [5:0] TEMP2;
	assign TEMP1 = MODE[0] ? reg1:reg12;
	assign TEMP2 = MODE[1] ? reg2:TEMP1;
	assign OUTPUT = MODE[2]? CURR_PC_IO:TEMP2;   \end{lstlisting}
\section{User Constraint File}
    \begin{lstlisting}[language={Verilog},title={top.ucf}]
NET "CLOCK"   LOC = C9;
NET "LED[0]"  LOC = F12;
NET "LED[1]"  LOC = E12;
NET "LED[2]"  LOC = E11;
NET "LED[3]"  LOC = F11;
NET "LED[4]"  LOC = C11;
NET "LED[5]"  LOC = D11;
NET "LED[6]"  LOC = E9;
NET "LED[7]"  LOC = F9;
NET "MODE[0]" LOC = L13;
NET "MODE[1]" LOC = L14;
NET "MODE[2]" LOC = H18;
NET "FAST" 	  LOC = N17;
NET "RESET"   LOC = K17 | IOSTANDARD = LVTTL | PULLDOWN;  \end{lstlisting}



\end{document}  