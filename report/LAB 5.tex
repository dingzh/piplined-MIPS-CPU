\documentclass[a4paper]{article}
\usepackage[CJKaddspaces ,CJKchecksingle ,CJKnumber]{xeCJK}
    \setCJKmainfont[BoldFont={Adobe Heiti Std},
    ItalicFont={Adobe Kaiti Std}]{Adobe Song Std}
    \setCJKsansfont{Adobe Heiti Std}
    \setCJKmonofont{Adobe Fangsong Std}
    \punctstyle{hangmobanjiao}
\usepackage{booktabs}
\usepackage{listings}
\usepackage{xcolor}


\lstset{ %
  backgroundcolor=\color{white},   % choose the background color; you must add \usepackage{color} or \usepackage{xcolor}
  basicstyle=\footnotesize,        % the size of the fonts that are used for the code
  breakatwhitespace=false,         % sets if automatic breaks should only happen at whitespace
  breaklines=true,                 % sets automatic line breaking
  captionpos=bl,                    % sets the caption-position to bottom
  deletekeywords={...},            % if you want to delete keywords from the given language
  escapeinside={\%*}{*)},          % if you want to add LaTeX within your code
  extendedchars=true,              % lets you use non-ASCII characters; for 8-bits encodings only, does not work with UTF-8
  frame=single,                    % adds a frame around the code
  keepspaces=true,                 % keeps spaces in text, useful for keeping indentation of code (possibly needs columns=flexible)
  keywordstyle=\color{blue},       % keyword style
  %language=Python,                 % the language of the code
  morekeywords={*,...},            % if you want to add more keywords to the set
  numbers=left,                    % where to put the line-numbers; possible values are (none, left, right)
  numbersep=5pt,                   % how far the line-numbers are from the code
  rulecolor=\color{black},         % if not set, the frame-color may be changed on line-breaks within not-black text (e.g. comments (green here))
  showspaces=false,                % show spaces everywhere adding particular underscores; it overrides 'showstringspaces'
  showstringspaces=false,          % underline spaces within strings only
  showtabs=false,                  % show tabs within strings adding particular underscores
  stepnumber=1,                    % the step between two line-numbers. If it's 1, each line will be numbered
  stringstyle=\color{orange},     % string literal style
  tabsize=2,                       % sets default tabsize to 2 spaces
}
\newcommand{\HRule}{\rule{\linewidth}{0.5mm}}

\begin{document}

\begin{titlepage}
    \begin{center} % Upper part of the page
    \includegraphics[width=0.15\textwidth]{sjtu.png}\\[1cm]
    \textsc{\LARGE CS334 计算机组成实验}\\[1.5cm]
    \textsc{\Large Lab6}\\[0.5cm] % Title
    \HRule \\[0.4cm] { \huge \bfseries 简单的类 MIPS单周期处理器}\\[0.4cm]
    \HRule \\[1.5cm] % Author and supervisor
    \begin{minipage}{0.4\textwidth}
        \begin{flushleft}
        \large
        周鼎\\
        5140219268
        \end{flushleft}
        \end{minipage}
        \begin{minipage}{0.4\textwidth}
        \begin{flushright}
        \large
        指导老师:王老师
        \end{flushright}
        \end{minipage}
        \vfill % Bottom of the page
        {\large \today}
    \end{center}
\end{titlepage}


\section{顶层结构}
            \begin{figure}[htbp]
                \centering
                    \includegraphics[scale = 0.8]{top.pdf}
                \caption{单周期MIPS顶层结构}\label{TOP}
            \end{figure}
            \begin{center}
              \textbf{注}:jump指令相应的data path将在后续添加。
            \end{center}



\section{PC}
    PC是一个十分简单然而特别关键的模块:它只需完成在时钟上升沿更新PC、在随后一个周期中保持的功能;它又是整个设计中程序能否正确执行和跳转的关键,要注意需要初始置零(也可通过运行时先加reset信号实现)。实现的代码和仿真结果见PC.v和Figure\ref{PCSIM}
    \begin{lstlisting}[language={Verilog},title={PC.v}]
module Pc(
    input clock_in,
    input [31:0] nextPC,
    output[31:0] currPC,
    input rst
    );	
	 reg [31:0] PCFile;
	 initial PCFile <= 0;

	 always @(posedge clock_in)	 begin
		if(rst) PCFile <= 0;
		else PCFile <= nextPC;
	 end
	
	 assign currPC = PCFile;

endmodule   \end{lstlisting}
    \begin{figure}[h]
    \centering
    \includegraphics[width=1\textwidth]{pcsim.png}
    \caption{PC行为仿真}\label{PCSIM}
    \end{figure}


\section{InstrMem}
    处理器执行的指令需要装载在InstrMen中,因其十分简单直接在Top中实现,不另设计模块。注意指令为32位,按word读取将PC右移2位。
    \begin{lstlisting}[language={Verilog},title={instrMem.v}]
    //define AND initial Instruction Memory
	reg [31:0] InstrMemory[9:0];
	initial $readmemb("./src/inst.txt",InstrMemory);
    wire [31:0] INSTR;
    wire [31:0] CURR_PC;
	assign INSTR = InstrMemory[CURR_PC>>2];  //fetch instruction by PC   \end{lstlisting}

\section{signExtender}
    符号扩展用于I-type指令扩展其后16位,可简单调用系统任务实现。
    \begin{lstlisting}[language={Verilog},title={signExtender.v}]
    wire [31:0] extRes;
    assign extRes = $signed(INSTR[15:0]);   \end{lstlisting}
\section{MUX}
    MUX可通``MUX=Sel\quad ?Opt1:Opt2''实现。
    \begin{lstlisting}[language={Verilog},title={MUXes.v}]
	assign MEM_REG_MUX = MEM_TO_REG ? MEM_READ_DATA : ALU_RES;	
	//R-type or load word
	assign REG_ALU_MUX = ALU_SRC ? EXTENDED_RES : REG_DATA_2;
	//which one to be used by ALU as the 2nd src
	assign REG_WRITE_ADDRESS = REG_DST ? INSTR[15:11] : INSTR[20:16];
	//which reg will be written   \end{lstlisting}

\section{nextPC}
    PC更新有三种情况:
    \begin{lstlisting}[language={Verilog}]
    PC <= PC + 4;
    PC <= (INSTR[15:0]<<2) + PC + 4;                //BEQ
    PC <= {INSTR[31:28],((PC + 4)<<2)[27:0]};       //JUMP    \end{lstlisting}
\begin{figure}
  \centering
  \includegraphics[width=1\textwidth]{nextpc.pdf}
  \caption{nextPC生成}
\end{figure}
\begin{lstlisting}[language={Verilog},title={nextPC.v}]
	wire [31:0] JUMP_ADDRESS;
	wire [31:0] BEQ_ADDRESS;
	//assign JUMP_ADDRESS
	assign JUMP_ADDRESS[31:28] = PCp4[31:28];
	assign JUMP_ADDRESS[27:2]  = INSTR[25:0];
	assign JUMP_ADDRESS[1:0]   = 'b00;
	//assign BEQ_ADDRESS
	assign BEQ_ADDRESS[31:0] = (EXTENDED_RES<<2) + PCp4;	
	
	wire [31:0] PC_SRC_SEL;
    wire [31:0] NEXT_PC;
	//wire [31:0] PCp4;
	//wire [31:0] CURR_PC;  //declared above;
	wire PC_SRC;
	assign PC_SRC = BRANCH & ZERO;	
	assign PCp4[31:0] = CURR_PC[31:0] + 4;
	//serval MUX below
	assign PC_SRC_SEL[31:0] = PC_SRC ? BEQ_ADDRESS[31:0]: PCp4[31:0];
	assign NEXT_PC[31:0] = JUMP ? JUMP_ADDRESS[31:0] : PC_SRC_SEL[31:0];   \end{lstlisting}

\section{信号线和模块实例化}
    对连线系统的命名有助于连线不重不漏,也便于实例化模块。
    \begin{lstlisting}[language={Verilog},title={wire.v}]
//                  main control signal                 //
	wire REG_DST,
		  JUMP,
		  BRANCH,
		  MEM_READ,
		  MEM_TO_REG,
		  MEM_WRITE,
		  ALU_SRC,
		  REG_WRITE,
		  ZERO;  //generated by ALU
	wire [1:0]  ALU_OP;
	wire [3:0]  ALU_CTR;
	
//                  data bus                    //

	wire [31:0] REG_DATA_1;
	wire [31:0] REG_DATA_2;
    //reg read output
	wire [4:0]  REG_WRITE_ADDRESS;
    //reg write address, the data will be from the MEM_REG_MUX
	wire [31:0] ALU_RES;
	wire [31:0] EXTENDED_RES;	
	wire [31:0] MEM_READ_DATA;	//address specified by alu result
	
	wire [31:0] MEM_REG_MUX;	
    //to be written into reg(memomy or alu result)
	wire [31:0] REG_ALU_MUX;
    //which one to be used by ALU as the 2nd src(rt or imm)   \end{lstlisting}
    \begin{lstlisting}[language={Verilog},title={instances.v}]
	Pc mainPC(
      .clock_in(CLOCK_IN),
      .nextPC(NEXT_PC),
      .currPC(CURR_PC),
		.rst(RESET)
    );	
	Ctr mainCtr(
		.opCode(INSTR[31:26]),
		.regDst(REG_DST),
		.aluSrc(ALU_SRC),
		.memToReg(MEM_TO_REG),
		.regWrite(REG_WRITE),
		.memRead(MEM_READ),
		.memWrite(MEM_WRITE),
		.branch(BRANCH),
		.aluOp(ALU_OP),
		.jump(JUMP)
	);	
	AluCtr mainAluCtr (
		.aluOp(ALU_OP),
		.funct(INSTR[5:0]),
		.aluCtr(ALU_CTR)
	);	
	Alu mainAlu (
		.input1(REG_DATA_1),
		.input2(REG_ALU_MUX),
		.aluCtr(ALU_CTR),
		.zero(ZERO),
		.aluRes(ALU_RES)
	);	
	signExt mainSignExt (
		.inst(INSTR[15:0]),
		.data(EXTENDED_RES)
	);	
	dataMemory mainDataMemory (
		.clock_in(CLOCK_IN),
		.address(ALU_RES),
		.writeData(REG_DATA_2),
		.readData(MEM_READ_DATA),
		.memWrite(MEM_WRITE),
		.memRead(MEM_READ)
	);	
	Register mainRegister (
		.clock_in(CLOCK_IN),
		.regWrite(REG_WRITE),
		.readReg1(INSTR[25:21]),
		.readReg2(INSTR[20:16]),
		.writeReg(REG_WRITE_ADDRESS),
		.writeData(MEM_REG_MUX),
		.reset(RESET),
		.readData1(REG_DATA_1),
		.readData2(REG_DATA_2),
		.reg1(reg1),
		.reg2(reg2)
	);	
endmodule   \end{lstlisting}

\section{ModelSim仿真}
    \begin{figure}[hb]
        \begin{minipage}[t]{0.6\linewidth}
        \centering
            \begin{lstlisting}[language={Verilog}]
00001000000000000000000000000010
00000000000000000000000000000000
10001100000000010000000000000000
10001100000000100000000000000100
00000000001000100000100000100000
00000000001000100001000000100000
00010000000000001111111111111101  \end{lstlisting}
        \caption{二进制指令}
        \end{minipage}%
        \begin{minipage}[t]{0.4\linewidth}
        \centering
            \begin{lstlisting}[language={Verilog}]
j 2
nop
lw	$1,	(0)$0;
lw	$2, (4)$0;
AGAIN:  ADD $1, $1, $2;
        ADD $2, $1, $2;
beq $0, $0, AGAIN;   \end{lstlisting}
        \caption{MIPS指令}
        \end{minipage}
    \end{figure}

    \begin{figure}[hb]
    \centering
        \includegraphics[width=1\textwidth]{topsim.png}
    \caption{ModelSim仿真波形}
    \end{figure}

\section{上板输入输出实现方案}
    \subsection{IO Scheme}
        \begin{figure}[h]
        \begin{minipage}[t]{0.51\linewidth}
        \centering
        \begin{tabular}{cc}
          \toprule
          SW[2:0] & MODE\\
          % after \\: \hline or \cline{col1-col2} \cline{col3-col4} ...
          \midrule
          000 & led[5:3]为\$2,led[2:0]为\$1 \\
          001 & led显示\$1 \\
          011 & led显示\$2 \\
          111 & led显示PC \\
          \bottomrule
        \end{tabular}
        \end{minipage}%
        \begin{minipage}[t]{0.49\linewidth}
        \centering
        \begin{tabular}{l}
        \qquad    设计使用8个LED作为输出,\\
        4个开关和1个复位式按钮作为输\\
        入,具体功能见下图。因板上接口\\
        限制,将只对输出寄存器\$1,\$2和\\
        PC,通过SW[2:0]选择输出模式,\\
        见左图。\\
        \end{tabular}

        \end{minipage}
    \end{figure}

    \begin{table}[h]
        \centering
        \begin{tabular}{ll}
              \toprule
              IO Port & Function\\
              \midrule
              LED[7]    &   查看clock信号 \\
              LED[6]    &   查看reset信号 \\
              LED[5:0]  &   查看寄存器数值 \\
              SW[3]     &   调整时钟快/慢模式 \\
              SW[2:0]   &   设置寄存器数值输出模式 \\
              Button    &   输入reste信号 \\
              \bottomrule
            \end{tabular}
            \caption{IO Scheme}

    \end{table}
    \subsection{IO控制逻辑}
    \begin{lstlisting}[language={Verilog},title={IO.v}]
////            GENRERATING SLOW CLOCK          ////
	wire CLOCK_IN;
	reg [26:0] Buffer = 0;
	always@ (posedge CLOCK) Buffer = Buffer + 1;
	assign CLOCK_IN = FAST ? CLOCK : Buffer[26];

////            IO MODE SEL                     ////
	assign LED[7] = RESET;
	assign LED[6] = CLOCK_IN;
	wire [5:0] OUTPUT;
	assign LED[5:0] = OUTPUT;
	wire [31:0] reg1;
	wire [31:0] reg2;
	wire [5:0]  reg12;
	assign reg12[5:3] = reg2;
	assign reg12[2:0] = reg1;
	wire [31:0] CURR_PC_IO;
  assign CURR_PC_IO = CURR_PC>>2;
	wire [5:0] TEMP1;
	wire [5:0] TEMP2;
	assign TEMP1 = MODE[0] ? reg1:reg12;
	assign TEMP2 = MODE[1] ? reg2:TEMP1;
	assign OUTPUT = MODE[2]? CURR_PC_IO:TEMP2;  \end{lstlisting}
    \subsection{User Constraint File}
    \begin{lstlisting}[language={Verilog},title={top.ucf}]
NET "CLOCK"   LOC = C9;
NET "LED[0]"  LOC = F12;
NET "LED[1]"  LOC = E12;
NET "LED[2]"  LOC = E11;
NET "LED[3]"  LOC = F11;
NET "LED[4]"  LOC = C11;
NET "LED[5]"  LOC = D11;
NET "LED[6]"  LOC = E9;
NET "LED[7]"  LOC = F9;
NET "MODE[0]" LOC = L13;
NET "MODE[1]" LOC = L14;
NET "MODE[2]" LOC = H18;
NET "FAST" 	  LOC = N17;
NET "RESET"   LOC = K17 | IOSTANDARD = LVTTL | PULLDOWN;  \end{lstlisting}


\end{document} 